\documentclass[11pt, oneside]{article}   	
\usepackage[margin=1in]{geometry}             		
\geometry{letterpaper}                   		
\usepackage{graphicx}					
\usepackage{amssymb}
\usepackage{amsmath}
\numberwithin{equation}{subsection}
\renewcommand{\thesection}{\arabic{section}.}
\renewcommand{\thesubsection}{-}
\usepackage{hyperref}
\usepackage{lastpage}
\usepackage{subcaption}
\usepackage{graphicx}
%Change me (below) to your graphics directory!!!
%\graphicspath{{electric_meter_data/}}
\graphicspath{{.}} 
\usepackage{xcolor,colortbl}

\usepackage{tabularx}

\usepackage{fancyhdr}

\usepackage[parfill]{parskip}
\setlength{\headheight}{15.2pt}


\fancypagestyle{plain}{%  the preset of fancyhdr 
\fancyhf{} % clear all header and footer fields
\fancyfoot[C]{\thepage} % except the center
\renewcommand{\headrulewidth}{0pt}
\renewcommand{\footrulewidth}{0pt}
\fancyhead[l]{Alex Coy, ac2456}
\fancyhead[c]{Running jacktrip for CUWS}
\fancyhead[r]{Page \thepage \hspace{1pt} of \pageref{LastPage}\qquad \qquad Summer 2020}
}



%These aren't used. I made this template though because I was bored and concerned that excel wouldn't format correctly.
\title{ECE Lab Template}
\author{Alex Coy}							

\fancyhead[l]{Alex Coy, ac2456}
\fancyhead[c]{Running jacktrip for CUWS}
\fancyhead[r]{Page \thepage \hspace{1pt} of \pageref{LastPage}\qquad \qquad Summer 2020}

\begin{document}
\pagestyle{fancy}

\section{Running Jacktrip}
At the moment, I have run jack and jacktrip as user processes. The difficulty
presented here is that user processes usually disappear upon logout or
connection loss. One common way around that is to use GNU screen. Simply
type
\begin{verbatim}
screen
\end{verbatim}
to begin a new screen session. To detach from a screen session (so you can
log off), you must first press Control-a, then press d. That will detach
you from your screen session, and a message will show that you detached.
To re-attach, use 
\begin{verbatim}
screen -r
\end{verbatim}

Jack is already installed for all server users. However, it is problematic
if jack is already running from another jacktrip admin; multiple jack servers
might confuse our computer. To check if jack is running, use:
\begin{verbatim}
ps -e | grep jack
\end{verbatim}
Jackd and jacktrip will appear if they are running, similar to below.
\begin{verbatim}
$ ps -e | grep jack
 9926 pts/2    00:00:00 jackd
\end{verbatim}
Server administrators can kill another user's jackd and jacktrip processes with
\begin{verbatim}
sudo pkill jacktrip
\end{verbatim}
or
\begin{verbatim}
sudo kill <PID>
\end{verbatim}
The PID of jackd is 9926 in the above ps reading.
If jackd is not running, first confirm that you're in a screen session.
Start jackd with
\begin{verbatim}
jackd -d dummy -r 48000 -p 64 &
\end{verbatim}
Start jacktrip with
\begin{verbatim}
jacktrip -S -z -q 12 -p 4
\end{verbatim}
During a production run, it is more useful to run with -p 2 so that participants
only hear the others. I will update this document with more strategies as we
find them.

\section{Other jack utilities}
I installed a few useful utilities. I advise running these in their own screen
sessions. You can give names to your screen sessions when starting them. See:
\begin{verbatim}
screen -S jacktrip
\end{verbatim}
You could detach and then re-attach with
\begin{verbatim}
screen -r jacktrip
\end{verbatim}
If you are in an existing screen session, you can ctrl-a then
": sessionname your\_name\_here" to rename the screen session.
\subsection{jack\_capture}
Jack\_capture is a very basic recording program. It doesn't look like there's
a way to make it capture jacktrip stuff automatically, so I installed another
program. Most likely, you'll want to start jack\_capture as
\begin{verbatim}
jack_capture --manual-connections -c <number>
\end{verbatim}
which sets up a recorder for $<$number$>$ channels of audio, which you can edit
later.

\subsection{njconnect}
Njconnect is a terminal-based connection-maker for jack. It's similar to
qjackctl (Jack Control on Windows), except it doesn't require any sort of
graphical interface. Just use "njconnect" to start it after jack and possibly
jacktrip and jack\_capture are running. Press "a" to see the server's audio
connections. Use the arrow keys to navigate inputs, outputs, and existing
connections. Highlight an input/output pair and press "c" to connect. Highlight
a connection and press "d" to disconnect. Press "q" to quit.

\end{document}  
